\documentclass[UTF8]{ctexart}
\usepackage{geometry, CJKutf8}
\geometry{margin=1.5cm, vmargin={0pt,1cm}}
\setlength{\topmargin}{-1cm}
\setlength{\paperheight}{29.7cm}
\setlength{\textheight}{25.3cm}




\begin{document}

李春阳, 学号3230101204

{数据结构与算法期末作业}

Dec.12th, 2024

\section{测试程序的设计思路}

编写了主程序main.cpp和含有各函数的头文件expression-evaluator.h

主程序中 通过输入一个四则运算算式 然后回车

首先通过expression-evaluator.h头文件中的format函数对输入的四则运算式进行格式化为str数组

对于str的每个字符进行判断 对于首位为负数,增加为0- 并将0-插入到str里

如果出现+-负数的情况 修正为 +0-负数 并将0-插入到负数前 

这样1+-2的此类情况格式化为 1+0-2 得以计算成功

对于科学计算法 将e转换为10的次方 如1+2e2 格式化为1+2*10*10 再插入str中 确保参与计算

format后运行Check-and-Preprocessing函数

该函数检查四则运算式是否格式正确

分别通过检查括号是否匹配

预处理负号

检查运算符是否合法

如果出现非法符号或者 运算符两侧不为数或括号 返回false

主程序显示ILLEAGL运算式格式错误 退出

正确则返回true 主程序进入solve函数进行表达式计算

表达式求值的思路主要是将中缀表达式转换为后缀表达式,

然后由后缀表达式进行求值,这里用到的数据结构主要是栈。

中缀转后缀:遍历表达式,如果是数字,就直接输出,

如果是操作符,就需要判断此时栈内是否为空或者栈顶是否为左括号

如果是的话,就直接进栈,否则就需要与栈顶元素进行比较

如果优先级大于栈顶元素,直接进栈,如果小于或等于,则需要先出栈,再进栈

当遍历完整个表达式之后,如果栈不为空,则依次出栈输出。

后缀求值:遍历后缀表达式,如果是数字,直接进栈

如果是操作符,则将栈顶两个元素出栈,进行运算

运算结果进行入栈,最终栈顶元素就是表达式的值

里面对除数为0的表达式 输出为ILLEAGL出现除零错误

测试了 大量四则运算 包括+ - * / 均运算正常

考虑了科学计数法 可以如 3+2e2 等四则运算 
 
考虑了 负数 可以完成1+-2的运算 但1--2认为是错误

测试例子 

1+2+(3*3) 结果为 12

1+-2.1 结果为-1.1

20+2e2 结果为220

1+2*3-6.5 结果为0.5

1+(2  结果为ILLEAGL运算式格式错误

1/0  结果为 ILLEAGL出现除零错误

我用 valgrind 进行测试,发现没有发生内存泄露。

\section{(可选)bug报告}

我没有发现 bug,:



\end{document}

%%% Local Variables: 
%%% mode: latex
%%% TeX-master: t
%%% End: 

