\documentclass[UTF8]{ctexart}
\usepackage{geometry, CJKutf8}
\geometry{margin=1.5cm, vmargin={0pt,1cm}}
\setlength{\topmargin}{-1cm}
\setlength{\paperheight}{29.7cm}
\setlength{\textheight}{25.3cm}

% useful packages.
\usepackage{amsfonts}
\usepackage{amsmath}
\usepackage{amssymb}
\usepackage{amsthm}
\usepackage{enumerate}
\usepackage{graphicx}
\usepackage{multicol}
\usepackage{fancyhdr}
\usepackage{layout}
\usepackage{listings}
\usepackage{float, caption}
\usepackage{pifont}

\lstset{
    basicstyle=\ttfamily, basewidth=0.5em
}

% some common command
\newcommand{\dif}{\mathrm{d}}
\newcommand{\avg}[1]{\left\langle #1 \right\rangle}
\newcommand{\difFrac}[2]{\frac{\dif #1}{\dif #2}}
\newcommand{\pdfFrac}[2]{\frac{\partial #1}{\partial #2}}
\newcommand{\OFL}{\mathrm{OFL}}
\newcommand{\UFL}{\mathrm{UFL}}
\newcommand{\fl}{\mathrm{fl}}
\newcommand{\op}{\odot}
\newcommand{\Eabs}{E_{\mathrm{abs}}}
\newcommand{\Erel}{E_{\mathrm{rel}}}

\begin{document}

\pagestyle{fancy}
\fancyhead{}
\lhead{李春阳, 3230101204}
\chead{第六次作业}
\rhead{Nov.11th, 2024}
\section{设计思路}
这段代码是一个二叉搜索树(Binary Search Tree, BST)的实现,它包含了删除节点、克隆树、获取树高、更新节点高度、右旋、左旋和平衡树的操作。

功能特点
模板类设计:通过模板参数Comparable,该BST类可以存储任何可比较的数据类型。
异常处理:定义了多个异常类,如UnderflowException和IllegalArgumentException,用于处理特定的错误情况。
自平衡功能:通过balance函数和相关的旋转操作(rightRotate和leftRotate),保持树的平衡,确保操作的效率。
拷贝和移动语义:提供了拷贝构造函数、移动构造函数、拷贝赋值运算符和移动赋值运算符,支持资源的有效管理。
递归和循环操作:结合递归和循环方法实现树的遍历和操作,提高效率。
内存管理:通过makeEmpty函数释放树中所有节点占用的内存,避免内存泄露。
主要函数和方法
构造函数和析构函数:初始化和销毁BST,包括拷贝和移动构造函数。
findMin和findMax:查找并返回树中的最小和最大元素。
contains:检查树中是否包含指定的元素。
isEmpty:检查树是否为空。
printTree:打印树的结构。
makeEmpty:清空树中的所有元素。
insert:插入一个元素到树中,支持常量引用和右值引用。
remove:从树中移除指定的元素。
clone:克隆整棵树,实现深拷贝。
section;测试结果测试结果正常,没有发生元素复制,没有段错误。
利用 valgrind 进行测试,没有发生内存泄露。



\end{document}

